\documentclass{beamer}
\mode<presentation>
{
  \usetheme{ldv}
  \setbeamercovered{transparent}
}

% Uncomment this if you're giving a presentation in german...
\usepackage[ngerman]{babel}

% ...and rename this to "Folie"
\newcommand{\slidenomenclature}{Folie}


\usepackage[utf8]{inputenc}
\usepackage{amsmath,amssymb,amsfonts}
\usepackage{times}
\usepackage{graphicx}
\usepackage{fancyvrb}
\usepackage{array}
\usepackage{colortbl}
\usepackage{tabularx}

% Uncomment me when you need to insert code
\usepackage{color}
\usepackage{listings}
\usepackage{minted}
\usepackage{algpseudocode}
% End Code

\usepackage{datetime}
\usepackage{tikz}

\usetikzlibrary{calc}
\usetikzlibrary{shapes.geometric}
\usetikzlibrary{decorations.pathreplacing}

% Uncomment me when you need video or sound
% \usepackage{multimedia}
% \usepackage{hyperref}
% End video

% Header
\newcommand{\zwischentitel}{Woche 5}
\newcommand{\leitthema}{Tobias Eppacher}
\newcommand{\presdatum}{\formatdate{26}{5}{2025}}
% End Header

% Titlepage
\title{Grundlagen: Algorithmen und Datenstrukturen}
\author{Tobias Eppacher}
\date{\presdatum}
\institute{School of Computation, Information and Technology}
\subtitle{Woche 5}
% End Titlepage


% Slides
\begin{document}


% 1. Slide: Titlepage
\begin{frame}
	\titlepage
\end{frame}

% 2. Slide: TOC
\begin{frame}
	\frametitle{Inhalt}
	\tableofcontents[subsectionstyle=hide]
\end{frame}

\section{Aufgaben}
\begin{frame}
	\frametitle{Aufgabe 5.1 - Laufzeitanalyse: Mergesort}

	In dieser Aufgabe machen wir eine Laufzeitanalyse von Mergesort auf drei verschiedene Weisen.

	\renewcommand{\theenumi}{\alph{enumi}}
	\begin{enumerate}
		\item \textbf{Argumentativ: } Wie viele Rekursionsebenen (ohne initialen Aufruf)? Asymptotischer Aufwand für jede Rekursionsebene?
		      Was ist damit der asymptotische Aufwand für den gesamten Algorithmus?
		\item \textbf{Iteratives Einsetzen}
		\item \textbf{Vollständige Induktion}
	\end{enumerate}

	\medskip

	Rekursive Formulierung der Laufzeit:
	\begin{align*}
		T(n) & = T(\lfloor\frac{n}{2}\rfloor) + T(\lceil \frac{n}{2} \rceil) + \Theta(n) \\
		T(1) & = \Theta(1)
	\end{align*}

\end{frame}

\begin{frame}
	\frametitle{Aufgabe 5.1 - Laufzeitanalyse: Mergesort (a)}
\end{frame}

\begin{frame}
	\frametitle{Aufgabe 5.1 - Laufzeitanalyse: Mergesort (b/c Anmerkungen)}

\end{frame}

\begin{frame}
	\frametitle{Aufgabe 5.1 - Laufzeitanalyse: Mergesort (b)}
\end{frame}

\begin{frame}
	\frametitle{Aufgabe 5.1 - Laufzeitanalyse: Mergesort (c)}
\end{frame}

\begin{frame}
	\frametitle{Aufgabe 5.2 - Betrunkener Übungsleiter}

	Wir betrachten einen torkelnden Übungsleiter an einer Kletterwand, der verschiedene Operationen ausführen kann.

	\bigskip

	Starthöhe: $h = 0$
	\renewcommand{\arraystretch}{1.3}
	\begin{table}[]
		\begin{tabular}{lll}
			Operation                        & Höhenänderung          & Laufzeit           \\
			\hline
			\mintinline{java}{hoch}          & $h = h + 1$            & $T = 1$            \\
			\mintinline{java}{runter(int k)} & $h = h - \min\{h, k\}$ & $T = \min\{h, k\}$
		\end{tabular}
	\end{table}

	{\small \textit{Anmerkung: \mintinline{java}{runter} verringert die Höhe nie unter $0$!}}

	\bigskip

	Zeigen Sie mithilfe der Bankkonto-Methode, dass die amortisierten Laufzeiten der Operationen
	\mintinline{java}{hoch} und \mintinline{java}{runter(int k)} in $\mathcal{O}(1)$ liegen.
\end{frame}

\begin{frame}
	\frametitle{Aufgabe 5.2 - Betrunkener Übungsleiter}
\end{frame}

\begin{frame}
	\frametitle{Aufgabe 5.2 - Betrunkener Übungsleiter}
\end{frame}

\begin{frame}[fragile]
	\frametitle{Aufgabe 5.3 - Stapelschlange}

	\begin{columns}

		\begin{column}{0.48\textwidth}
			\begin{minted}[fontsize=\scriptsize]{java}
class Stapelschlange {
  private Stack s1 = new Stack();
  private Stack s2 = new Stack();

  public void enqueue(int v) {
  	s1.push(v);
  }
  
  public int dequeue() {
    if(s2.isEmpty())
      while (!s1.isEmpty())
        s2.push(s1.pop());
    return s2.pop();
  }
}
			\end{minted}

			{\scriptsize
			\begin{table}
				\begin{tabular}{|l|l|}
					\hline
					Stack-Methode     & Laufzeit         \\ \hline
					void push(int v)  & $\mathcal{O}(1)$ \\ \hline
					int pop()         & $\mathcal{O}(1)$ \\ \hline
					boolean isEmpty() & $\mathcal{O}(1)$ \\ \hline
				\end{tabular}
			\end{table}
			}
		\end{column}

		\begin{column}{0.48\textwidth}
			\renewcommand{\theenumi}{\alph{enumi}}
			\begin{enumerate}
				\item Was ist die Worst-Case Laufzeit von \mintinline{java}{dequeue} (abhängig von Queuegröße $n$)?
				\item Finden Sie ein Amortisierungsschema, das die amortisierte Laufzeit der Operationen minimiert. \\
				      Geben Sie die Laufzeiten an und zeigen Sie die Korrektheit des Schemas?
			\end{enumerate}
		\end{column}
	\end{columns}
\end{frame}

\begin{frame}
	\frametitle{Aufgabe 5.3 - Stapelschlange (a)}
\end{frame}

\begin{frame}
	\frametitle{Aufgabe 5.3 - Stapelschlange (b)}
\end{frame}

\begin{frame}
	\frametitle{Aufgabe 5.3 - Stapelschlange (b)}
\end{frame}

\section{E-Aufgaben}
\begin{frame}
	\frametitle{E-Aufgaben}
	\begin{itemize}
		\item Aufgabe 5.4 - Instabile Sortierverfahren \\
		      \begin{itemize}
			      \item Worstcase Suche für verschiedene Pivot-Wahlen
		      \end{itemize}
		\item Aufgabe 5.5 - Datenstruktur Amore \\
		      \begin{itemize}
			      \item Amortisierte Analyse
		      \end{itemize}
	\end{itemize}
\end{frame}

\section{Hausaufgaben}
\begin{frame}
	\frametitle{Hausaufgaben}
	\begin{itemize}
		\item Hausaufgabe 3 - Dynamisches Array \\
		      (Deadline: 28.05.2025)
		\item Hausaufgabe 4 - Verbessertes Sortieren \\
		      (Deadline: 28.05.2025)
		\item Hausaufgabe 5 - Radixsort \\
		      (Deadline: 04.06.2025)
	\end{itemize}
\end{frame}

\begin{frame}
	\textbf{Fragen?}
	\begin{itemize}
		\item Nach Übung gerne bei mir melden
		\item Tutoriumschannel oder DM an mich auf Zulip
		\item Vorlesungschannels von GAD auf Zulip (insbesondere bei Hausaufgaben)
	\end{itemize}

	\medskip
	\textbf{Feedback oder Verbesserungsvorschläge?} \\
	Gerne nach dem Tutorium mit mir quatschen oder DM auf Zulip

	\medskip
	\textbf{Bis nächste Woche!}
\end{frame}

% End Slides

\end{document}
