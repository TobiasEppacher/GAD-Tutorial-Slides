\begin{frame}
    \frametitle{Aufgabe 10.1 - Hashing mit Chaining}
    \scriptsize
    Veranschaulichen Sie Hashing mit Chaining. Die Größe $m$ der Hash-Tabelle ist in den folgenden Beispielen jeweils die Primzahl $11$. Die folgenden Operationen sollen nacheinander ausgeführt werden:
    \begin{itemize}
      \item insert $3, 11, 9, 7, 14, 56, 4, 12, 15, 8, 1$
      \item delete $56$
      \item insert $25$
    \end{itemize}
    Der Einfachheit halber sollen die Schlüssel der Elemente die Elemente selbst sein.
  
    \medskip
  
    a) Verwenden Sie zunächst die Hashfunktion
    $$g(x) = 5x \mod 11$$
  
    b) Berechnen Sie die Hashwerte unter Verwendung der Hashfunktion
    $$h(x) = \mathbf{a} \cdot \mathbf{x} \mod m$$
  
    nach dem aus der Vorlesung bekannten Verfahren füreinfache universelle Hashfunktionen,
    wobei $\mathbf{a} = (7, 5)$ und $\mathbf{x} = (\lfloor \frac{x}{2^w} \rfloor \mod 2^w, x \mod 2^w)$
    für $w = \lfloor \log_2 m \rfloor = \lfloor 3.45 \dots \rfloor = 3$ gilt
    und der Ausdruck $\mathbf{a} \cdot \mathbf{x}$ ein Skalarprodukt bezeichnet.
  
  \end{frame}
  
  \begin{frame}[t]
    \frametitle{Aufgabe 10.1 - Hashing mit Chaining (a)}
  
    $$g(x) = 5x \mod 11$$
  
    \bigskip
    \bigskip
    \bigskip
    \bigskip
    \bigskip
    \bigskip
    \bigskip
    \bigskip
    \bigskip
    \bigskip
  
    \begin{center}
      \begin{tabular}{c|c|c|c|c|c|c|c|c|c|c|c|c}
        $k(e)$    & 1 & 3 & 4 & 7 & 8 & 9 & 11 & 12 & 14 & 15 & 25 & 56 \\
        \hline
        $g(k(e))$ &   &   &   &   &   &   &    &    &    &    &    &    \\
      \end{tabular}
    \end{center}
  \end{frame}
  
  \begin{frame}
    \frametitle{Aufgabe 10.1 - Hashing mit Chaining (a)}
  
    \begin{center}
      \begin{tabular}{c|c|c|c|c|c|c|c|c|c|c|c|c}
        $k(e)$    & 1 & 3 & 4 & 7 & 8 & 9 & 11 & 12 & 14 & 15 & 25 & 56 \\
        \hline
        $g(k(e))$ & 5 & 4 & 9 & 2 & 7 & 1 & 0  & 5  & 4  & 9  & 4  & 5  \\
      \end{tabular}
    \end{center}
  \end{frame}
  
  \begin{frame}
    \frametitle{Aufgabe 10.1 - Hashing mit Chaining (a)}
  
    1. Operation: insert(3)
  
    \begin{center}
      \begin{tabular}{c|c|c|c|c|c|c|c|c|c|c|c|c}
        $k(e)$    & 1 & 3 & 4 & 7 & 8 & 9 & 11 & 12 & 14 & 15 & 25 & 56 \\
        \hline
        $g(k(e))$ & 5 & 4 & 9 & 2 & 7 & 1 & 0  & 5  & 4  & 9  & 4  & 5  \\
      \end{tabular}
  
      \bigskip
  
      \begin{tabular}{c|c|c|c|c|c|c|c|c|c|c}
        0 & 1 & 2 & 3 & 4 & 5 & 6 & 7 & 8 & 9 & 10 \\
        \hline
          &   &   &   &   &   &   &   &   &   &    \\
          &   &   &   &   &   &   &   &   &   &    \\
          &   &   &   &   &   &   &   &   &   &    \\
      \end{tabular}
    \end{center}
  \end{frame}
  
  \begin{frame}
    \frametitle{Aufgabe 10.1 - Hashing mit Chaining (a)}
    2. Operation: insert(11)
    \begin{center}
      \begin{tabular}{c|c|c|c|c|c|c|c|c|c|c|c|c}
        $k(e)$    & 1 & 3 & 4 & 7 & 8 & 9 & 11 & 12 & 14 & 15 & 25 & 56 \\
        \hline
        $g(k(e))$ & 5 & 4 & 9 & 2 & 7 & 1 & 0  & 5  & 4  & 9  & 4  & 5  \\
      \end{tabular}
  
      \bigskip
  
      \begin{tabular}{c|c|c|c|c|c|c|c|c|c|c}
        0 & 1 & 2 & 3 & 4 & 5 & 6 & 7 & 8 & 9 & 10 \\
        \hline
          &   &   &   & 3 &   &   &   &   &   &    \\
          &   &   &   &   &   &   &   &   &   &    \\
          &   &   &   &   &   &   &   &   &   &    \\
      \end{tabular}
    \end{center}
  \end{frame}
  
  \begin{frame}
    \frametitle{Aufgabe 10.1 - Hashing mit Chaining (a)}
    3. Operation: insert(9)
    \begin{center}
      \begin{tabular}{c|c|c|c|c|c|c|c|c|c|c|c|c}
        $k(e)$    & 1 & 3 & 4 & 7 & 8 & 9 & 11 & 12 & 14 & 15 & 25 & 56 \\
        \hline
        $g(k(e))$ & 5 & 4 & 9 & 2 & 7 & 1 & 0  & 5  & 4  & 9  & 4  & 5  \\
      \end{tabular}
  
      \bigskip
  
      \begin{tabular}{c|c|c|c|c|c|c|c|c|c|c}
        0  & 1 & 2 & 3 & 4 & 5 & 6 & 7 & 8 & 9 & 10 \\
        \hline
        11 &   &   &   & 3 &   &   &   &   &   &    \\
           &   &   &   &   &   &   &   &   &   &    \\
           &   &   &   &   &   &   &   &   &   &    \\
      \end{tabular}
    \end{center}
  \end{frame}
  
  \begin{frame}
    \frametitle{Aufgabe 10.1 - Hashing mit Chaining (a)}
    4. Operation: insert(7)
    \begin{center}
      \begin{tabular}{c|c|c|c|c|c|c|c|c|c|c|c|c}
        $k(e)$    & 1 & 3 & 4 & 7 & 8 & 9 & 11 & 12 & 14 & 15 & 25 & 56 \\
        \hline
        $g(k(e))$ & 5 & 4 & 9 & 2 & 7 & 1 & 0  & 5  & 4  & 9  & 4  & 5  \\
      \end{tabular}
  
      \bigskip
  
      \begin{tabular}{c|c|c|c|c|c|c|c|c|c|c}
        0  & 1 & 2 & 3 & 4 & 5 & 6 & 7 & 8 & 9 & 10 \\
        \hline
        11 & 9 &   &   & 3 &   &   &   &   &   &    \\
           &   &   &   &   &   &   &   &   &   &    \\
           &   &   &   &   &   &   &   &   &   &    \\
      \end{tabular}
    \end{center}
  \end{frame}
  
  \begin{frame}
    \frametitle{Aufgabe 10.1 - Hashing mit Chaining (a)}
    5. Operation: insert(14)
    \begin{center}
      \begin{tabular}{c|c|c|c|c|c|c|c|c|c|c|c|c}
        $k(e)$    & 1 & 3 & 4 & 7 & 8 & 9 & 11 & 12 & 14 & 15 & 25 & 56 \\
        \hline
        $g(k(e))$ & 5 & 4 & 9 & 2 & 7 & 1 & 0  & 5  & 4  & 9  & 4  & 5  \\
      \end{tabular}
  
      \bigskip
  
      \begin{tabular}{c|c|c|c|c|c|c|c|c|c|c}
        0  & 1 & 2 & 3 & 4 & 5 & 6 & 7 & 8 & 9 & 10 \\
        \hline
        11 & 9 & 7 &   & 3 &   &   &   &   &   &    \\
           &   &   &   &   &   &   &   &   &   &    \\
           &   &   &   &   &   &   &   &   &   &    \\
      \end{tabular}
    \end{center}
  \end{frame}
  
  \begin{frame}
    \frametitle{Aufgabe 10.1 - Hashing mit Chaining (a)}
    6. Operation: insert(56)
    \begin{center}
      \begin{tabular}{c|c|c|c|c|c|c|c|c|c|c|c|c}
        $k(e)$    & 1 & 3 & 4 & 7 & 8 & 9 & 11 & 12 & 14 & 15 & 25 & 56 \\
        \hline
        $g(k(e))$ & 5 & 4 & 9 & 2 & 7 & 1 & 0  & 5  & 4  & 9  & 4  & 5  \\
      \end{tabular}
  
      \bigskip
  
      \begin{tabular}{c|c|c|c|c|c|c|c|c|c|c}
        0  & 1 & 2 & 3 & 4  & 5 & 6 & 7 & 8 & 9 & 10 \\
        \hline
        11 & 9 & 7 &   & 3  &   &   &   &   &   &    \\
           &   &   &   & 14 &   &   &   &   &   &    \\
           &   &   &   &    &   &   &   &   &   &    \\
      \end{tabular}
    \end{center}
  \end{frame}
  
  \begin{frame}
    \frametitle{Aufgabe 10.1 - Hashing mit Chaining (a)}
    7. Operation: insert(4)
    \begin{center}
      \begin{tabular}{c|c|c|c|c|c|c|c|c|c|c|c|c}
        $k(e)$    & 1 & 3 & 4 & 7 & 8 & 9 & 11 & 12 & 14 & 15 & 25 & 56 \\
        \hline
        $g(k(e))$ & 5 & 4 & 9 & 2 & 7 & 1 & 0  & 5  & 4  & 9  & 4  & 5  \\
      \end{tabular}
  
      \bigskip
  
      \begin{tabular}{c|c|c|c|c|c|c|c|c|c|c}
        0  & 1 & 2 & 3 & 4  & 5  & 6 & 7 & 8 & 9 & 10 \\
        \hline
        11 & 9 & 7 &   & 3  & 56 &   &   &   &   &    \\
           &   &   &   & 14 &    &   &   &   &   &    \\
           &   &   &   &    &    &   &   &   &   &    \\
      \end{tabular}
    \end{center}
  \end{frame}
  
  \begin{frame}
    \frametitle{Aufgabe 10.1 - Hashing mit Chaining (a)}
    8. Operation: insert(12)
    \begin{center}
      \begin{tabular}{c|c|c|c|c|c|c|c|c|c|c|c|c}
        $k(e)$    & 1 & 3 & 4 & 7 & 8 & 9 & 11 & 12 & 14 & 15 & 25 & 56 \\
        \hline
        $g(k(e))$ & 5 & 4 & 9 & 2 & 7 & 1 & 0  & 5  & 4  & 9  & 4  & 5  \\
      \end{tabular}
  
      \bigskip
  
      \begin{tabular}{c|c|c|c|c|c|c|c|c|c|c}
        0  & 1 & 2 & 3 & 4  & 5  & 6 & 7 & 8 & 9 & 10 \\
        \hline
        11 & 9 & 7 &   & 3  & 56 &   &   &   & 4 &    \\
           &   &   &   & 14 &    &   &   &   &   &    \\
           &   &   &   &    &    &   &   &   &   &    \\
      \end{tabular}
    \end{center}
  \end{frame}
  
  \begin{frame}
    \frametitle{Aufgabe 10.1 - Hashing mit Chaining (a)}
    9. Operation: insert(15)
    \begin{center}
      \begin{tabular}{c|c|c|c|c|c|c|c|c|c|c|c|c}
        $k(e)$    & 1 & 3 & 4 & 7 & 8 & 9 & 11 & 12 & 14 & 15 & 25 & 56 \\
        \hline
        $g(k(e))$ & 5 & 4 & 9 & 2 & 7 & 1 & 0  & 5  & 4  & 9  & 4  & 5  \\
      \end{tabular}
  
      \bigskip
  
      \begin{tabular}{c|c|c|c|c|c|c|c|c|c|c}
        0  & 1 & 2 & 3 & 4  & 5  & 6 & 7 & 8 & 9 & 10 \\
        \hline
        11 & 9 & 7 &   & 3  & 56 &   &   &   & 4 &    \\
           &   &   &   & 14 & 12 &   &   &   &   &    \\
           &   &   &   &    &    &   &   &   &   &    \\
      \end{tabular}
    \end{center}
  \end{frame}
  
  \begin{frame}
    \frametitle{Aufgabe 10.1 - Hashing mit Chaining (a)}
    10. Operation: insert(8)
    \begin{center}
      \begin{tabular}{c|c|c|c|c|c|c|c|c|c|c|c|c}
        $k(e)$    & 1 & 3 & 4 & 7 & 8 & 9 & 11 & 12 & 14 & 15 & 25 & 56 \\
        \hline
        $g(k(e))$ & 5 & 4 & 9 & 2 & 7 & 1 & 0  & 5  & 4  & 9  & 4  & 5  \\
      \end{tabular}
  
      \bigskip
  
      \begin{tabular}{c|c|c|c|c|c|c|c|c|c|c}
        0  & 1 & 2 & 3 & 4  & 5  & 6 & 7 & 8 & 9  & 10 \\
        \hline
        11 & 9 & 7 &   & 3  & 56 &   &   &   & 4  &    \\
           &   &   &   & 14 & 12 &   &   &   & 15 &    \\
           &   &   &   &    &    &   &   &   &    &    \\
      \end{tabular}
    \end{center}
  \end{frame}
  
  \begin{frame}
    \frametitle{Aufgabe 10.1 - Hashing mit Chaining (a)}
    11. Operation: insert(1)
    \begin{center}
      \begin{tabular}{c|c|c|c|c|c|c|c|c|c|c|c|c}
        $k(e)$    & 1 & 3 & 4 & 7 & 8 & 9 & 11 & 12 & 14 & 15 & 25 & 56 \\
        \hline
        $g(k(e))$ & 5 & 4 & 9 & 2 & 7 & 1 & 0  & 5  & 4  & 9  & 4  & 5  \\
      \end{tabular}
  
      \bigskip
  
      \begin{tabular}{c|c|c|c|c|c|c|c|c|c|c}
        0  & 1 & 2 & 3 & 4  & 5  & 6 & 7 & 8 & 9  & 10 \\
        \hline
        11 & 9 & 7 &   & 3  & 56 &   & 8 &   & 4  &    \\
           &   &   &   & 14 & 12 &   &   &   & 15 &    \\
           &   &   &   &    &    &   &   &   &    &    \\
      \end{tabular}
    \end{center}
  \end{frame}
  
  \begin{frame}
    \frametitle{Aufgabe 10.1 - Hashing mit Chaining (a)}
    12. Operation: delete(56)
    \begin{center}
      \begin{tabular}{c|c|c|c|c|c|c|c|c|c|c|c|c}
        $k(e)$    & 1 & 3 & 4 & 7 & 8 & 9 & 11 & 12 & 14 & 15 & 25 & 56 \\
        \hline
        $g(k(e))$ & 5 & 4 & 9 & 2 & 7 & 1 & 0  & 5  & 4  & 9  & 4  & 5  \\
      \end{tabular}
  
      \bigskip
  
      \begin{tabular}{c|c|c|c|c|c|c|c|c|c|c}
        0  & 1 & 2 & 3 & 4  & 5  & 6 & 7 & 8 & 9  & 10 \\
        \hline
        11 & 9 & 7 &   & 3  & 56 &   & 8 &   & 4  &    \\
           &   &   &   & 14 & 12 &   &   &   & 15 &    \\
           &   &   &   &    & 1  &   &   &   &    &    \\
      \end{tabular}
    \end{center}
  \end{frame}
  
  \begin{frame}
    \frametitle{Aufgabe 10.1 - Hashing mit Chaining (a)}
    13. Operation: insert(25)
    \begin{center}
      \begin{tabular}{c|c|c|c|c|c|c|c|c|c|c|c|c}
        $k(e)$    & 1 & 3 & 4 & 7 & 8 & 9 & 11 & 12 & 14 & 15 & 25 & 56 \\
        \hline
        $g(k(e))$ & 5 & 4 & 9 & 2 & 7 & 1 & 0  & 5  & 4  & 9  & 4  & 5  \\
      \end{tabular}
  
      \bigskip
  
      \begin{tabular}{c|c|c|c|c|c|c|c|c|c|c}
        0  & 1 & 2 & 3 & 4  & 5  & 6 & 7 & 8 & 9  & 10 \\
        \hline
        11 & 9 & 7 &   & 3  & 12 &   & 8 &   & 4  &    \\
           &   &   &   & 14 & 1  &   &   &   & 15 &    \\
           &   &   &   &    &    &   &   &   &    &    \\
      \end{tabular}
    \end{center}
  \end{frame}
  
  \begin{frame}
    \frametitle{Aufgabe 10.1 - Hashing mit Chaining (a)}
    13. Operation: insert(25)
    \begin{center}
      \begin{tabular}{c|c|c|c|c|c|c|c|c|c|c|c|c}
        $k(e)$    & 1 & 3 & 4 & 7 & 8 & 9 & 11 & 12 & 14 & 15 & 25 & 56 \\
        \hline
        $g(k(e))$ & 5 & 4 & 9 & 2 & 7 & 1 & 0  & 5  & 4  & 9  & 4  & 5  \\
      \end{tabular}
  
      \bigskip
  
      \begin{tabular}{c|c|c|c|c|c|c|c|c|c|c}
        0  & 1 & 2 & 3 & 4  & 5  & 6 & 7 & 8 & 9  & 10 \\
        \hline
        11 & 9 & 7 &   & 3  & 12 &   & 8 &   & 4  &    \\
           &   &   &   & 14 & 1  &   &   &   & 15 &    \\
           &   &   &   & 25 &    &   &   &   &    &    \\
      \end{tabular}
    \end{center}
  \end{frame}
  
  \begin{frame}
    \frametitle{Aufgabe 10.1 - Hashing mit Chaining (b)}
    b) Berechnen Sie die Hashwerte unter Verwendung der Hashfunktion
    $$h(x) = \mathbf{a} \cdot \mathbf{x} \mod m$$
    nach dem aus der Vorlesung bekannten Verfahren für einfache universelle Hashfunktionen,
    wobei $\mathbf{a} = (7, 5)$ und $\mathbf{x} = (\lfloor \frac{x}{2^w} \rfloor \mod 2^w, x \mod 2^w)$
    für $w = \lfloor \log_2 m \rfloor = \lfloor 3.45 \dots \rfloor = 3$ gilt
    und der Ausdruck $\mathbf{a} \cdot \mathbf{x}$ ein Skalarprodukt bezeichnet.
  
    \bigskip
    \bigskip
  
    \begin{center}
      \begin{tabular}{c|c|c|c|c|c|c|c|c|c|c|c|c}
        $k(e)$    & 1 & 3 & 4 & 7 & 8 & 9 & 11 & 12 & 14 & 15 & 25 & 56 \\
        \hline
        $h(k(e))$ &   &   &   &   &   &   &    &    &    &    &    &    \\
      \end{tabular}
    \end{center}
  \end{frame}
  
  \begin{frame}[t]
    \frametitle{Aufgabe 10.1 - Hashing mit Chaining (b)}
    $$h(x) = \mathbf{a} \cdot \mathbf{x} \mod m$$
    $$\mathbf{a} = (7, 5) \qquad \mathbf{x} = \left(\lfloor \frac{x}{2^w} \rfloor \mod 2^w, x \mod 2^w\right) $$
  
    \bigskip
    \bigskip
    \bigskip
    \bigskip
    \bigskip
    \bigskip
    \bigskip
    \bigskip
  
    \begin{center}
      \begin{tabular}{c|c|c|c|c|c|c|c|c|c|c|c|c}
        $k(e)$    & 1 & 3 & 4 & 7 & 8 & 9 & 11 & 12 & 14 & 15 & 25 & 56 \\
        \hline
        $h(k(e))$ &   &   &   &   &   &   &    &    &    &    &    &    \\
      \end{tabular}
    \end{center}
  \end{frame}
  
  \begin{frame}
    \frametitle{Aufgabe 10.1 - Hashing mit Chaining (b)}
    \begin{center}
      \begin{tabular}{c|c|c|c|c|c|c|c|c|c|c|c|c}
        $k(e)$    & 1 & 3 & 4 & 7 & 8 & 9 & 11 & 12 & 14 & 15 & 25 & 56 \\
        \hline
        $h(k(e))$ & 5 & 4 & 9 & 2 & 7 & 1 & 0  & 5  & 4  & 9  & 4  & 5  \\
      \end{tabular}
    \end{center}
  \end{frame}