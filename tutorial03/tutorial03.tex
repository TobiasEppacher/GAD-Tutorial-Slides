\documentclass{beamer}
\mode<presentation>
{
  \usetheme{ldv}
  \setbeamercovered{transparent}
}

% Uncomment this if you're giving a presentation in german...
\usepackage[ngerman]{babel}

% ...and rename this to "Folie"
\newcommand{\slidenomenclature}{Folie}


\usepackage[utf8]{inputenc}
\usepackage{amsmath,amssymb,amsfonts}
\usepackage{times}
\usepackage{graphicx}
\usepackage{fancyvrb}
\usepackage{array}
\usepackage{colortbl}
\usepackage{tabularx}

% Uncomment me when you need to insert code
\usepackage{color}
\usepackage{listings}
\usepackage{minted}
\usepackage{algpseudocode}
% End Code

\usepackage{datetime}

% Uncomment me when you need video or sound
% \usepackage{multimedia}
% \usepackage{hyperref}
% End video

% Header
\newcommand{\zwischentitel}{Woche 3}
\newcommand{\leitthema}{Tobias Eppacher}
\newcommand{\presdatum}{\formatdate{12}{5}{2025}}
% End Header

% Titlepage
\title{Grundlagen: Algorithmen und Datenstrukturen}
\author{Tobias Eppacher}
\date{\presdatum}
\institute{School of Computation, Information and Technology}
\subtitle{Woche 3}
% End Titlepage


% Slides
\begin{document}

% 1. Slide: Titlepage
\begin{frame}
	\titlepage
\end{frame}

% 2. Slide: TOC
\begin{frame}
	\frametitle{Table of contents}
	\tableofcontents[subsectionstyle=hide]
\end{frame}

% Further Slides
\section{Aufgaben}

\subsection{Zufallsvariablen}
\begin{frame}
	\frametitle{Aufgabe 3.1 - Zufallsvariablen}
	\textbf{Binäre Zufallsvariablen} \\
	\begin{itemize}
		\item Binäre Zufallsvariable $Y$
		\item Wert von $Y$ nur $0$ oder $1$
		\item In dieser Aufgabe: konstante Wahrscheinlichkeiten \\
		      $\mathbb{P}(Y=1) = c$ und $\mathbb{P}(Y=0)=1-c \quad c \in [0, 1]$
		\item Erwartungswert: $\mathbb{E}[Y]=\sum_{y \in 0, 1} y * \mathbb{P}[Y=y]$ \\
		      $\mathbb{E}[Y]=0 * \mathbb{P}[Y=0] + 1 * \mathbb{P}[Y=1] = c$
	\end{itemize}
	\medskip
	\textit{Erwartungswert = Summe der möglichen Werte gewichtet mit deren Auftrittswahrscheinlichkeit}
\end{frame}

\begin{frame}
	\frametitle{Aufgabe 3.1 - Zufallsvariablen}

	Gegeben sei eine Zahl $a \in {0, 1}$ sowie eine Zahlenfolge $(x_1, x_2, . . . , x_k)$, wobei $x_i \in {0, 1}$
	für alle $i \in {1, . . . , k}$ gilt, wobei $k$ die Länge der Zahlenfolge ist. Gefragt ist, ob $a$ in der
	Zahlenfolge vorkommt.

	\begin{columns}[T]

		\begin{column}{.50\textwidth}
			\textbf{Input:} int $a$, int[] $(x_1, x_2, \dots, x_k)$
			\begin{algorithmic}[1]
				\State{int $i := 1$}
				\While{$i \leq k$}
				\If{$x_{i} = a$}
				\State{\textbf{return} Ja}
				\EndIf{}
				\State{$i := i + 1$}
				\EndWhile{}
			\end{algorithmic}
		\end{column}
		\begin{column}{.48\textwidth}
			\begin{enumerate}
				\item Erwartete Anzahl Vergleiche $x_i = a$ \\
				      (\textit{Alle Eingaben gleich wahrscheinlich})
				\item Asymptotisch erwartete Laufzeit
			\end{enumerate}
			\medskip
			$\sum_{i=0}^{n} c^i = \frac{c^{n+1}-1}{c-1} \quad c \neq 1$
		\end{column}
	\end{columns}
\end{frame}

\begin{frame}[t]
	\frametitle{Aufgabe 3.1 - Zufallsvariablen}
	Was ist die erwartete Anzahl an Vergleichen ($x_i = a$) bei gleicher Wahrscheinlichkeit für alle Eingaben?
\end{frame}

\begin{frame}[t]
	\frametitle{Aufgabe 3.1 - Zufallsvariablen}
	Was ist die asymptotisch erwartete Laufzeit?
\end{frame}

\subsection{Komplexitätsanalyse}
\begin{frame}[fragile]
	\frametitle{Aufgabe 3.2 - Komplexitätsanalyse}
	Gegeben sei eine Zahlenfolge $A[0], \dots, A[n - 1]$, wobei die Länge $n$ ist, und eine Zahl $x$ aus
	dieser Folge. \\
	Gehen Sie im Folgenden davon aus, dass die Wahrscheinlichkeit von Duplikaten
	in $A$ vernachlässigbar gering ist, das heißt ihre Anzahl ist im Durchschnitt in $\mathcal{O}(1)$.

	\medskip
	\begin{minted}[fontsize=\footnotesize]{c}
// count(x, A)
int c = 0;
for (int i = 0; i < n; i++)
	if (A[i] == x) c++;
return c;
	\end{minted}
\end{frame}

\begin{frame}[t, fragile]
	\frametitle{Aufgabe 3.2 - Komplexitätsanalyse (a)}
	Bestimmen Sie die Komplexität von \mintinline{java}{count} im worst-case, best-case und average-case.
	\medskip
	\begin{minted}[fontsize=\footnotesize]{c}
// count(x, A)
int c = 0;
for (int i = 0; i < n; i++)
	if (A[i] == x) c++;
return c;
	\end{minted}
\end{frame}

\begin{frame}[t]
	\frametitle{Aufgabe 3.2 - Komplexitätsanalyse (b)}
	Überlegen Sie sich einen alternativen Algorithmus, der auf \textbf{sortierten} Folgen arbeitet,
	und bestimmen Sie dessen Komplexität im worst-case, best-case und average-case.
	Vergleichen Sie diese mit den Ergebnissen aus Aufgabenteil a).
\end{frame}

\begin{frame}[t, fragile]
	\frametitle{Aufgabe 3.2 - Komplexitätsanalyse (b)}
	\begin{minted}[fontsize=\footnotesize]{c}
// count(x, A) mit early stopping
int c = 0;
for (int i = 0; i < n; i++) {
	if (A[i] == x) { c++ };
	else { if (A[i] > x) return c; }
}
return c;
	\end{minted}
\end{frame}

\begin{frame}[t, fragile]
	\frametitle{Aufgabe 3.2 - Komplexitätsanalyse (b)}
	\begin{minted}[fontsize=\scriptsize]{c}
// count(x, A) mit binärer Suche 
int c = 0, l = 0, r = n - 1, m;
while (l <= r) {
  m = (l + r) / 2;
  if (A[m] == x) {
    while (m != -1 && A[m] == x) {
      m--; c++;
    }
    m = (l + r) / 2 + 1;
    while (m != A.length && A[m] == x) {
      m++; c++;
    }
    return c;
  }
  if (A[m] < x) l = m + 1;
  else r = m - 1;
}
return c;
	\end{minted}
\end{frame}

\begin{frame}[t]
	\frametitle{Aufgabe 3.2 - Komplexitätsanalyse (b)}
	Andere Vorschläge? (wenn genug Zeit)
\end{frame}

\begin{frame}[t]
	\frametitle{Aufgabe 3.2 - Komplexitätsanalyse (c)}
	Lässt sich die Komplexität verbessern, indem man die Folge zunächst sortiert?
\end{frame}


\subsection{Selbstorganisierende Liste}
\begin{frame}[fragile]
	\frametitle{Aufgabe 3.3 - Selbstorganisierende Liste}
	\begin{minted}[fontsize=\scriptsize]{java}
public class SelfOrganizingList<T> {
  private static class Node<T> {
    T data;
    Node<T> next;

    Node(T d) {
      data = d;
      next = null;
    }
  }

  public void add(T data) {
    // TODO: Neuer Knoten am Listenende + O(1) Laufzeit
  }

  public Optional<T> findFirst(Predicate<T> p) {
    // TODO: return das erste p.test(n.data) == true
  }

  public void removeDuplicates() {
    // TODO: Behalte erste Vorkommen von Elementen + O(1) Speicher
  }
}
	\end{minted}
\end{frame}

\begin{frame}[t, fragile]
	\frametitle{Aufgabe 3.3 - Selbstorganisierende Liste (a)}
	\begin{minted}[fontsize=\scriptsize]{java}
public void add(T data) {



















}
	\end{minted}
\end{frame}

\begin{frame}[t, fragile]
	\frametitle{Aufgabe 3.3 - Selbstorganisierende Liste (b)}
	\begin{minted}[fontsize=\scriptsize]{java}
public Optional<T> findFirst(Predicate<T> p) {



















}
	\end{minted}
\end{frame}

\begin{frame}[t, fragile]
	\frametitle{Aufgabe 3.3 - Selbstorganisierende Liste (c)}
	\begin{minted}[fontsize=\scriptsize]{java}
public void removeDuplicates() {



















}
	\end{minted}
\end{frame}

\subsection{Aufgabe 3.4 - Algorithmen und Laufzeiten}
\begin{frame}
	\frametitle{Aufgabe 3.4 - Algorithmen und Laufzeiten}
	Entwerfen Sie im Folgenden zwei einfache Funktionen und erarbeiten Sie korrekten \mintinline{java}{java}-Code. \\
	Bestimmen Sie dann die asymptotische Laufzeit Ihres Algorithmus. \\
	Sie dürfen in ihren Algorithmen ausschließlich Schleifen, If-Statements, Additionen, Subtraktion sowie Multiplikation verwenden.

	\medskip
	\renewcommand{\theenumi}{\alph{enumi}}
	\begin{enumerate}
		\item Ganzzahldivision von zwei \mintinline{java}{int} $a > 0$ und $b > 0$
		\item Rest für Division (Modulo) von zwei \mintinline{java}{int} $a > 0$ und $b \neq 0$
	\end{enumerate}
\end{frame}

\begin{frame}[t]
	\frametitle{Aufgabe 3.4 - Algorithmen und Laufzeiten (a)}
	Ganzzahldivision von zwei \mintinline{java}{int} $a > 0$ und $b > 0$
\end{frame}

\begin{frame}[t]
	\frametitle{Aufgabe 3.4 - Algorithmen und Laufzeiten (b)}
	Rest für Division (Modulo) von zwei \mintinline{java}{int} $a > 0$ und $b \neq 0$
\end{frame}

\section{E-Aufgaben}
\begin{frame}
	\frametitle{E-Aufgaben}
	\begin{itemize}
		\item Aufgabe 3.5 - Noch mehr Spaß mit $\mathcal{O}$ \\
		      \begin{itemize}
			      \item Gute Überprüfung für Verständnis von $\mathcal{O}$
			      \item Kleiner Induktionsbeweis
		      \end{itemize}
		\item Aufgabe 3.6 - Zufallsvariablen-Caching
		      \begin{itemize}
			      \item Übung zu Zufallsvariablen
		      \end{itemize}
	\end{itemize}
\end{frame}

\section{Hausaufgaben}
\begin{frame}
	\frametitle{Hausaufgaben}
	\begin{itemize}
		\item Hausaufgabe 3 - Dynamisches Array \\
		      (Deadline: 21.05.2025)
	\end{itemize}
\end{frame}

\begin{frame}
	\textbf{Fragen?}
	\begin{itemize}
		\item Nach Übung gerne bei mir melden
		\item Tutoriumschannel oder DM an mich auf Zulip
		\item Vorlesungschannels von GAD auf Zulip (insbesondere bei Hausaufgaben)
	\end{itemize}

	\medskip
	\textbf{Feedback oder Verbesserungsvorschläge?} \\
	Gerne nach dem Tutorium mit mir quatschen oder DM auf Zulip

	\medskip
	\textbf{Bis nächste Woche!}
\end{frame}

% End Slides

\end{document}
