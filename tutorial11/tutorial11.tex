\documentclass{beamer}
\mode<presentation>
{
  \usetheme{ldv}
  \setbeamercovered{transparent}
}

% Uncomment this if you're giving a presentation in german...
\usepackage[ngerman]{babel}

% ...and rename this to "Folie"
\newcommand{\slidenomenclature}{Folie}


\usepackage[utf8]{inputenc}
\usepackage{amsmath,amssymb,amsfonts}
\usepackage{times}
\usepackage{graphicx}
\usepackage{fancyvrb}
\usepackage{array}
\usepackage{colortbl}
\usepackage{tabularx}

% Uncomment me when you need to insert code
\usepackage{color}
\usepackage{listings}
\usepackage{minted}
\usepackage{algpseudocode}
% End Code

\usepackage{datetime}
\usepackage{tikz}

\usetikzlibrary{calc}
\usetikzlibrary{shapes.geometric}
\usetikzlibrary{decorations.pathreplacing}

% Uncomment me when you need video or sound
% \usepackage{multimedia}
% \usepackage{hyperref}
% End video

% Header
\newcommand{\zwischentitel}{Woche 11}
\newcommand{\leitthema}{Tobias Eppacher}
\newcommand{\presdatum}{\formatdate{7}{7}{2025}}
% End Header

% Titlepage
\title{Grundlagen: Algorithmen und Datenstrukturen}
\author{Tobias Eppacher}
\date{\presdatum}
\institute{School of Computation, Information and Technology}
\subtitle{Woche 11}
% End Titlepage


% Slides
\begin{document}


% 1. Slide: Titlepage
\begin{frame}
	\titlepage
\end{frame}

% 2. Slide: TOC
\begin{frame}
	\frametitle{Inhalt}
	\tableofcontents[subsectionstyle=hide]
\end{frame}

\section{Aufgaben}

\begin{frame}
	\frametitle{Aufgabe 11.1 - Universelles Hashing}

	\small
	Die Pinguingattungen {Brillenpinguin, Zwergpinguin, Eselspinguin, Kaiserpinguin, Goldschopfpinguin}
	sollen in einer Hash-Tabelle der Größe $m = 4$ untergebracht werden. Es seien folgende Hashfunktionen gegeben:

	\begin{table}
		\tiny
		\centering
		\begin{tabular}{llllll}
			$f_1:$ & Brillenp. $\mapsto 4$ & Zwergp. $\mapsto 2$ & Eselsp. $\mapsto 2$ & Kaiserp. $\mapsto 1$ & Goldschopfp. $\mapsto 4$ \\
			$f_2:$ & Brillenp. $\mapsto 3$ & Zwergp. $\mapsto 4$ & Eselsp. $\mapsto 2$ & Kaiserp. $\mapsto 3$ & Goldschopfp. $\mapsto 4$ \\
			$f_3:$ & Brillenp. $\mapsto 2$ & Zwergp. $\mapsto 2$ & Eselsp. $\mapsto 4$ & Kaiserp. $\mapsto 1$ & Goldschopfp. $\mapsto 1$ \\
			$f_4:$ & Brillenp. $\mapsto 1$ & Zwergp. $\mapsto 3$ & Eselsp. $\mapsto 3$ & Kaiserp. $\mapsto 4$ & Goldschopfp. $\mapsto 4$ \\
			\hline\hline
			$g_1:$ & Brillenp. $\mapsto 1$ & Zwergp. $\mapsto 1$ & Eselsp. $\mapsto 3$ & Kaiserp. $\mapsto 2$ & Goldschopfp. $\mapsto 3$ \\
			$g_2:$ & Brillenp. $\mapsto 2$ & Zwergp. $\mapsto 4$ & Eselsp. $\mapsto 2$ & Kaiserp. $\mapsto 3$ & Goldschopfp. $\mapsto 4$ \\
			$g_3:$ & Brillenp. $\mapsto 4$ & Zwergp. $\mapsto 4$ & Eselsp. $\mapsto 1$ & Kaiserp. $\mapsto 4$ & Goldschopfp. $\mapsto 2$ \\
			$g_4:$ & Brillenp. $\mapsto 3$ & Zwergp. $\mapsto 1$ & Eselsp. $\mapsto 2$ & Kaiserp. $\mapsto 3$ & Goldschopfp. $\mapsto 3$ \\
			$g_5:$ & Brillenp. $\mapsto 4$ & Zwergp. $\mapsto 2$ & Eselsp. $\mapsto 2$ & Kaiserp. $\mapsto 2$ & Goldschopfp. $\mapsto 3$ \\
		\end{tabular}
	\end{table}

	In der Vorlesung haben wir den Begriff der $c$-universellen Hashfunktionen kennengelernt.

	\begin{enumerate}
		\renewcommand{\theenumi}{\alph{enumi}}
		\item Geben Sie für die Familie $\mathcal{H}_1 = \{ f_1, f_2, f_3, f_4 \}$ das kleinste $c$ an, so dass $\mathcal{H}_1$ $c$-universell ist
		\item Finden Sie eine möglichst kleine Familie $\mathcal{H}_2 \subseteq \{ g_1, g_2, g_3, g_4, g_5 \}$, die $1$-universell ist.
		      Untermauern Sie Ihre Aussagen mit glaubwürdigen Argumenten.
	\end{enumerate}
\end{frame}

\begin{frame}[t]
	\frametitle{Aufgabe 11.1 - Universelles Hashing (a)}
	\small
	Eine Familie von Hashfunktionen $\mathcal{H}$ ist $c$-universell, wenn für alle Wertepaare $x, y$ mit $x \neq y$ gilt:

	$$\frac{|\{ f \in \mathcal{H} : f(x) = f(y) \}|}{\mathcal{H}} \leq \frac{c}{m}$$

	\begin{table}
		\tiny
		\centering
		\begin{tabular}{|l||c|c|c|c||c|c|c|c|c|}
			\hline
			Paar                             & $f_1$ & $f_2$ & $f_3$ & $f_4$ & $g_1$ & $g_2$ & $g_3$ & $g_4$ & $g_5$ \\
			\hline\hline
			Brillenpinguin/Zwergpinguin      &       &       &       &       &       &       &       &       &       \\
			\hline
			Brillenpinguin/Eselspinguin      &       &       &       &       &       &       &       &       &       \\
			\hline
			Brillenpinguin/Kaiserpinguin     &       &       &       &       &       &       &       &       &       \\
			\hline
			Brillenpinguin/Goldschopfpinguin &       &       &       &       &       &       &       &       &       \\
			\hline
			Zwergpinguin/Eselspinguin        &       &       &       &       &       &       &       &       &       \\
			\hline
			Zwergpinguin/Kaiserpinguin       &       &       &       &       &       &       &       &       &       \\
			\hline
			Zwergpinguin/Goldschopfpinguin   &       &       &       &       &       &       &       &       &       \\
			\hline
			Eselspinguin/Kaiserpinguin       &       &       &       &       &       &       &       &       &       \\
			\hline
			Eselspinguin/Goldschopfpinguin   &       &       &       &       &       &       &       &       &       \\
			\hline
			Kaiserpinguin/Goldschopfpinguin  &       &       &       &       &       &       &       &       &       \\
			\hline
		\end{tabular}
	\end{table}
\end{frame}

\begin{frame}[t]
	\frametitle{Aufgabe 11.1 - Universelles Hashing (a)}
	\scriptsize
	$$\frac{|\{ f \in \mathcal{H} : f(x) = f(y) \}|}{\mathcal{H}} \leq \frac{c}{m}$$
	\begin{table}
		\tiny
		\centering
		\begin{tabular}{|l||c|c|c|c||c|c|c|c|c|}
			\hline
			Paar                             & $f_1$ & $f_2$ & $f_3$ & $f_4$ & $g_1$ & $g_2$ & $g_3$ & $g_4$ & $g_5$ \\
			\hline\hline
			Brillenpinguin/Zwergpinguin      &       &       & x     &       & x     &       & x     &       &       \\
			\hline
			Brillenpinguin/Eselspinguin      &       &       &       &       &       & x     &       &       &       \\
			\hline
			Brillenpinguin/Kaiserpinguin     &       & x     &       &       &       &       & x     & x     &       \\
			\hline
			Brillenpinguin/Goldschopfpinguin & x     &       &       &       &       &       &       & x     &       \\
			\hline
			Zwergpinguin/Eselspinguin        & x     &       &       & x     &       &       &       &       & x     \\
			\hline
			Zwergpinguin/Kaiserpinguin       &       &       &       &       &       &       & x     &       & x     \\
			\hline
			Zwergpinguin/Goldschopfpinguin   &       & x     &       &       &       & x     &       &       &       \\
			\hline
			Eselspinguin/Kaiserpinguin       &       &       &       &       &       &       &       &       & x     \\
			\hline
			Eselspinguin/Goldschopfpinguin   &       &       &       &       & x     &       &       &       &       \\
			\hline
			Kaiserpinguin/Goldschopfpinguin  &       &       & x     & x     &       &       &       & x     &       \\
			\hline
		\end{tabular}
	\end{table}

	Wir suchen kleinst-mögliches $c$ für $\mathcal{H}_1 = \{ f_1, f_2, f_3, f_4 \}$
\end{frame}

\begin{frame}
	\frametitle{Aufgabe 11.1 - Universelles Hashing (a) - Extraplatz}
\end{frame}

\begin{frame}[t]
	\frametitle{Aufgabe 11.1 - Universelles Hashing (b)}

	\scriptsize
	$$\frac{|\{ f \in \mathcal{H} : f(x) = f(y) \}|}{\mathcal{H}} \leq \frac{c}{m}$$
	\begin{table}
		\tiny
		\centering
		\begin{tabular}{|l||c|c|c|c||c|c|c|c|c|}
			\hline
			Paar                             & $f_1$ & $f_2$ & $f_3$ & $f_4$ & $g_1$ & $g_2$ & $g_3$ & $g_4$ & $g_5$ \\
			\hline\hline
			Brillenpinguin/Zwergpinguin      &       &       & x     &       & x     &       & x     &       &       \\
			\hline
			Brillenpinguin/Eselspinguin      &       &       &       &       &       & x     &       &       &       \\
			\hline
			Brillenpinguin/Kaiserpinguin     &       & x     &       &       &       &       & x     & x     &       \\
			\hline
			Brillenpinguin/Goldschopfpinguin & x     &       &       &       &       &       &       & x     &       \\
			\hline
			Zwergpinguin/Eselspinguin        & x     &       &       & x     &       &       &       &       & x     \\
			\hline
			Zwergpinguin/Kaiserpinguin       &       &       &       &       &       &       & x     &       & x     \\
			\hline
			Zwergpinguin/Goldschopfpinguin   &       & x     &       &       &       & x     &       &       &       \\
			\hline
			Eselspinguin/Kaiserpinguin       &       &       &       &       &       &       &       &       & x     \\
			\hline
			Eselspinguin/Goldschopfpinguin   &       &       &       &       & x     &       &       &       &       \\
			\hline
			Kaiserpinguin/Goldschopfpinguin  &       &       & x     & x     &       &       &       & x     &       \\
			\hline
		\end{tabular}
	\end{table}

	Wir suchen eine möglichst kleine Familie $\mathcal{H}_2 \subseteq \{ g_1, f_2, g_3, g_4, g_5 \}$, sodass $c = 1$ ist.
\end{frame}

\begin{frame}
	\frametitle{Aufgabe 11.1 - Universelles Hashing (b) - Extraplatz}
\end{frame}

\begin{frame}
	\frametitle{Aufgabe 11.2 - Die perfektes Hashtabelle}
	\small
	Konstruieren Sie eine statische perfekte Hashtabelle für die Elemente:
	$$(16, 10, 11) \  (8, 2, 15) \ (7, 12, 8) \  (1, 10, 3) \ (13, 11, 14) \ (6, 11, 14)$$
	$$(7, 3, 16) \  (2, 2, 8) \  (10, 5, 15)  \ (7, 3, 14)  \ (2, 10, 1)  \ (14, 11, 6)$$

	\begin{columns}
		\begin{column}{0.48\textwidth}
			Jedes Element $x$ besteht aus den Stellen $(x_0, x_1, x_2)$. Verwenden Sie jeweils passend eine der Hashfunktionen:
			\tiny
			\begin{gather*}
				\left( \sum_{i=0}^{2} 2^i x_i \right) \mod 17 \\
				\left( \sum_{i=0}^{2} a_i x_i \right) \mod 7 \text{ mit } a = (0, 0, 1) \text{ oder } a = (6, 6, 2) \\
				\left( \sum_{i=0}^{2} a_i x_i \right) \mod 3 \text{ mit } a = (1, 0, 0) \text{ oder } a = (0, 2, 2)
			\end{gather*}
		\end{column}
		\begin{column}{0.48\textwidth}
			\textbf{Stufe 1:} Ziehe aus $c$-universeller Familie $H_{\lceil \sqrt{2}cn \rceil}$ bis $C(n) < \sqrt{2}n$

			\medskip
			\textbf{Stufe 2:} Bei $b_\ell$ Elementen wähle Bucketgröße $m_\ell = cb_\ell(b_\ell - 1) + 1$. \\
			Ziehe aus $c$-universeller Familie $H_{m_\ell}$ bis injektiver Abbildung.

			\medskip
			\scriptsize
			\textit{Wir müssen hier nicht ziehen, sondern nur die angegebenen Hashfunktionen verwenden.}
		\end{column}
	\end{columns}
\end{frame}

\begin{frame}
	\frametitle{Aufgabe 11.2 - Die perfektes Hashtabelle}

	\begin{columns}
		\begin{column}{0.6\textwidth}
		\end{column}
		\begin{column}{0.38\textwidth}
			\begin{table}
				\centering
				\small
				\begin{tabular}{c|c}
					Element        & Bucket \\
					\hline
					$(7, 3, 14)$   & $1$    \\
					$(2, 2, 8)$    & $4$    \\
					$(8, 2, 15)$   & $4$    \\
					$(13, 11, 14)$ & $6$    \\
					$(2, 10, 1)$   & $9$    \\
					$(14, 11, 6)$  & $9$    \\
					$(7, 3, 16)$   & $9$    \\
					$(16, 10, 11)$ & $12$   \\
					$(7, 12, 8)$   & $12$   \\
					$(10, 5, 15)$  & $12$   \\
					$(1, 10, 3)$   & $16$   \\
					$(6, 11, 14)$  & $16$   \\
				\end{tabular}
			\end{table}
		\end{column}
	\end{columns}
\end{frame}

\begin{frame}
	\frametitle{Aufgabe 11.2 - Die perfektes Hashtabelle}

\end{frame}

\begin{frame}
	\frametitle{Aufgabe 11.2 - Die perfektes Hashtabelle}
	\small
	Finale Platzierungen der Elemente in der Hashtabelle:
	\begin{table}
		\centering
		\small
		\begin{tabular}{c|c|c}
			Element        & Bucket &     \\
			\hline
			$(7, 3, 14)$   & $1$    & $0$ \\
			$(2, 2, 8)$    & $4$    & $2$ \\
			$(8, 2, 15)$   & $4$    & $1$ \\
			$(13, 11, 14)$ & $6$    & $0$ \\
			$(2, 10, 1)$   & $9$    & $1$ \\
			$(14, 11, 6)$  & $9$    & $6$ \\
			$(7, 3, 16)$   & $9$    & $2$ \\
			$(16, 10, 11)$ & $12$   & $3$ \\
			$(7, 12, 8)$   & $12$   & $4$ \\
			$(10, 5, 15)$  & $12$   & $1$ \\
			$(1, 10, 3)$   & $16$   & $1$ \\
			$(6, 11, 14)$  & $16$   & $0$ \\
		\end{tabular}
	\end{table}
\end{frame}

\begin{frame}
	\frametitle{Aufgabe 11.3 Linear Probing}
	\scriptsize
	Veranschaulichen Sie Hashing mit Linear Probing.

	\smallskip
	Die Größe der Hashtabelle ist dabei jeweils $m = 13$. Führen Sie die folgenden Operationen aus:
	\begin{align*}
		 & \text{insert} \quad 16, 3, 12, 17, 29, 10, 24   \\
		 & \text{delete} \quad 16                          \\
		 & \text{insert} \quad 5, 1,                    15 \\
		 & \text{delete} \quad 10                          \\
		 & \text{insert} \quad 14                          \\
		 & \text{delete} \quad 1
	\end{align*}
	Verwenden Sie die Hashfunktion
	$$h(x) = 3x \mod 13$$

	Beim Löschen soll die dritte Methode aus der Vorlesung verwendet werden, d.h. die Wiederherstellung der folgenden Invariante: Für jedes Element $e$ in der Hashtabelle mit Schlüssel
	$k(e)$, aktueller Position $j$ und optimaler Position $i = h(k(e))$ sind alle Positionen $i,(i + 1) \mod m,(i + 2) \mod m, . . . , j$ der Hashtabelle belegt.
	Bei dieser Aufgabe soll keine dynamische Größenanpassung der Hashtabelle stattfinden.
\end{frame}

\begin{frame}
	\frametitle{Aufgabe 11.3 Linear Probing}
	1. Operation: \textbf{insert}(16) mit opt. Position: \textbf{9}
	\begin{table}
		\centering
		\begin{tabular}{c|c|c|c|c|c|c|c|c|c|c|c|c}
			0 & 1 & 2 & 3 & 4 & 5 & 6 & 7 & 8 & 9 & 10 & 11 & 12 \\
			\hline
			  &   &   &   &   &   &   &   &   &   &    &    &    \\
		\end{tabular}
	\end{table}

	2. Operation: \textbf{insert}(3) mit opt. Position: \textbf{9}
	\begin{table}
		\centering
		\begin{tabular}{c|c|c|c|c|c|c|c|c|c|c|c|c}
			0 & 1 & 2 & 3 & 4 & 5 & 6 & 7 & 8 & 9 & 10 & 11 & 12 \\
			\hline
			  &   &   &   &   &   &   &   &   &   &    &    &    \\
		\end{tabular}
	\end{table}

	3. Operation: \textbf{insert}(12) mit opt. Position: \textbf{10}
	\begin{table}
		\centering
		\begin{tabular}{c|c|c|c|c|c|c|c|c|c|c|c|c}
			0 & 1 & 2 & 3 & 4 & 5 & 6 & 7 & 8 & 9 & 10 & 11 & 12 \\
			\hline
			  &   &   &   &   &   &   &   &   &   &    &    &    \\
		\end{tabular}
	\end{table}
\end{frame}

\begin{frame}
	\frametitle{Aufgabe 11.3 Linear Probing}
	4. Operation: \textbf{insert}(17) mit opt. Position: \textbf{12}
	\begin{table}
		\centering
		\begin{tabular}{c|c|c|c|c|c|c|c|c|c|c|c|c}
			0 & 1 & 2 & 3 & 4 & 5 & 6 & 7 & 8 & 9 & 10 & 11 & 12 \\
			\hline
			  &   &   &   &   &   &   &   &   &   &    &    &    \\
		\end{tabular}
	\end{table}

	5. Operation: \textbf{insert}(29) mit opt. Position: \textbf{9}
	\begin{table}
		\centering
		\begin{tabular}{c|c|c|c|c|c|c|c|c|c|c|c|c}
			0 & 1 & 2 & 3 & 4 & 5 & 6 & 7 & 8 & 9 & 10 & 11 & 12 \\
			\hline
			  &   &   &   &   &   &   &   &   &   &    &    &    \\
		\end{tabular}
	\end{table}

	6. Operation: \textbf{insert}(10) mit opt. Position: \textbf{4}
	\begin{table}
		\centering
		\begin{tabular}{c|c|c|c|c|c|c|c|c|c|c|c|c}
			0 & 1 & 2 & 3 & 4 & 5 & 6 & 7 & 8 & 9 & 10 & 11 & 12 \\
			\hline
			  &   &   &   &   &   &   &   &   &   &    &    &    \\
		\end{tabular}
	\end{table}
\end{frame}

\begin{frame}
	\frametitle{Aufgabe 11.3 Linear Probing}
	7. Operation: \textbf{insert}(24) mit opt. Position: \textbf{7}
	\begin{table}
		\centering
		\begin{tabular}{c|c|c|c|c|c|c|c|c|c|c|c|c}
			0 & 1 & 2 & 3 & 4 & 5 & 6 & 7 & 8 & 9 & 10 & 11 & 12 \\
			\hline
			  &   &   &   &   &   &   &   &   &   &    &    &    \\
		\end{tabular}
	\end{table}

	8. Operation: \textbf{delete}(16) mit opt. Position: \textbf{9}
	\begin{table}
		\centering
		\begin{tabular}{c|c|c|c|c|c|c|c|c|c|c|c|c}
			0 & 1 & 2 & 3 & 4 & 5 & 6 & 7 & 8 & 9 & 10 & 11 & 12 \\
			\hline
			  &   &   &   &   &   &   &   &   &   &    &    &    \\
		\end{tabular}
	\end{table}

	9. Operation: \textbf{insert}(5) mit opt. Position: \textbf{2}
	\begin{table}
		\centering
		\begin{tabular}{c|c|c|c|c|c|c|c|c|c|c|c|c}
			0 & 1 & 2 & 3 & 4 & 5 & 6 & 7 & 8 & 9 & 10 & 11 & 12 \\
			\hline
			  &   &   &   &   &   &   &   &   &   &    &    &    \\
		\end{tabular}
	\end{table}
\end{frame}

\begin{frame}
	\frametitle{Aufgabe 11.3 Linear Probing}
	10. Operation: \textbf{insert}(1) mit opt. Position: \textbf{3}
	\begin{table}
		\centering
		\begin{tabular}{c|c|c|c|c|c|c|c|c|c|c|c|c}
			0 & 1 & 2 & 3 & 4 & 5 & 6 & 7 & 8 & 9 & 10 & 11 & 12 \\
			\hline
			  &   &   &   &   &   &   &   &   &   &    &    &    \\
		\end{tabular}
	\end{table}

	11. Operation: \textbf{insert}(15) mit opt. Position: \textbf{6}
	\begin{table}
		\centering
		\begin{tabular}{c|c|c|c|c|c|c|c|c|c|c|c|c}
			0 & 1 & 2 & 3 & 4 & 5 & 6 & 7 & 8 & 9 & 10 & 11 & 12 \\
			\hline
			  &   &   &   &   &   &   &   &   &   &    &    &    \\
		\end{tabular}
	\end{table}

	12. Operation: \textbf{delete}(10) mit opt. Position: \textbf{4}
	\begin{table}
		\centering
		\begin{tabular}{c|c|c|c|c|c|c|c|c|c|c|c|c}
			0 & 1 & 2 & 3 & 4 & 5 & 6 & 7 & 8 & 9 & 10 & 11 & 12 \\
			\hline
			  &   &   &   &   &   &   &   &   &   &    &    &    \\
		\end{tabular}
	\end{table}
\end{frame}

\begin{frame}
	\frametitle{Aufgabe 11.3 Linear Probing}
	13. Operation: \textbf{insert}(14) mit opt. Position: \textbf{3}
	\begin{table}
		\centering
		\begin{tabular}{c|c|c|c|c|c|c|c|c|c|c|c|c}
			0 & 1 & 2 & 3 & 4 & 5 & 6 & 7 & 8 & 9 & 10 & 11 & 12 \\
			\hline
			  &   &   &   &   &   &   &   &   &   &    &    &    \\
		\end{tabular}
	\end{table}

	14. Operation: \textbf{delete}(1) mit opt. Position: \textbf{3}
	\begin{table}
		\centering
		\begin{tabular}{c|c|c|c|c|c|c|c|c|c|c|c|c}
			0 & 1 & 2 & 3 & 4 & 5 & 6 & 7 & 8 & 9 & 10 & 11 & 12 \\
			\hline
			  &   &   &   &   &   &   &   &   &   &    &    &    \\
		\end{tabular}
	\end{table}
\end{frame}

\begin{frame}
	\frametitle{Aufgabe 11.4 - Double Hashing}
	\scriptsize
	Doppel-Hashing ist eine Methode zur Kollisionsbehandlung. Bei Kollisionen kommt eine
	Sondierungsfunktion zum Einsatz, die eine sekundäre Hashfunktion beinhaltet:
	$$s(x, i) = i \cdot h_2(x), i \in \mathbb{N}_0$$

	Diese Sondierungsfunktion wird angewendet, falls der durch die primare Hashfunktion $h_1(x)$
	berechnete Index bereits besetzt ist. Dabei wird $i$ beginnend bei $0$ bei jedem Versuch um $1$
	erhöht. Die vollständige Hashfunktion lautet dann:
	$$h(x, i) = (h_1(x) + s(x, i)) \mod m$$

	Verwenden Sie im Folgenden die Hashfunktionen
	\begin{align*}
		 & h_1(x) = (3x + 1) \mod m      \\
		 & h_2(x) = 1 + (x \mod (m - 1))
	\end{align*}

	a) Geben Sie die vollständige Hashfunktion $h(x, i)$ für eine Tabelle der Länge $m = 13$ an. \\
	b) Veranschaulichen Sie schrittweise das Einfügen der Schlüssel $12, 23, 13, 56, 26$ $45, 24, 94, 42$ in eine Hashtabelle der Länge $m = 13$.
\end{frame}

\begin{frame}[t]
	\frametitle{Aufgabe 11.4 - Double Hashing (a)}
	\small
	$$h(x, i) = (h_1(x) + s(x, i)) \mod m$$
	$$s(x, i) = i \cdot h_2(x), i \in \mathbb{N}_0$$
	\begin{gather*}
		h_1(x) = (3x + 1) \mod m      \\
		h_2(x) = 1 + (x \mod (m - 1))
	\end{gather*}
	$$\text{Mit } m = 13$$
\end{frame}

\begin{frame}
	\frametitle{Aufgabe 11.4 - Double Hashing (b)}
	\begin{table}
		\centering
		\small
		\begin{tabular}{c|c|c|c|c|c|c|c|c|c|c|c|c|c}
			          & $0$ & $1$ & $2$ & $3$ & $4$ & $5$ & $6$ & $7$ & $8$ & $9$ & $10$ & $11$ & $12$ \\
			\hline
			ins($12$) &     &     &     &     &     &     &     &     &     &     &      &      &      \\
			\hline
			ins($23$) &     &     &     &     &     &     &     &     &     &     &      &      &      \\
			\hline
			ins($13$) &     &     &     &     &     &     &     &     &     &     &      &      &      \\
			\hline
			ins($56$) &     &     &     &     &     &     &     &     &     &     &      &      &      \\
			\hline
			ins($26$) &     &     &     &     &     &     &     &     &     &     &      &      &      \\
			\hline
			ins($45$) &     &     &     &     &     &     &     &     &     &     &      &      &      \\
			\hline
			ins($24$) &     &     &     &     &     &     &     &     &     &     &      &      &      \\
			\hline
			ins($94$) &     &     &     &     &     &     &     &     &     &     &      &      &      \\
			\hline
			ins($42$) &     &     &     &     &     &     &     &     &     &     &      &      &      \\
			\hline
		\end{tabular}
	\end{table}
\end{frame}

\section{E-Aufgaben}
\begin{frame}
	\frametitle{E-Aufgaben}
	\begin{itemize}
		\item Aufgabe 11.5 - Bad Hash \\
		      \begin{itemize}
			      \item Intuition zu Hashfunktion
		      \end{itemize}
		\item Aufgabe 11.6 - Linear Probing \\
		      \begin{itemize}
			      \item Abgeänderte Aufgabe aus Tutorium
		      \end{itemize}
	\end{itemize}
\end{frame}

\section{Hausaufgaben}
\begin{frame}
	\frametitle{Hausaufgaben}
	\begin{itemize}
		\item Hausaufgabe 9 - AB-Baum \\
		      (Deadline: 09.07.2025)
		\item Hausaufgabe 10 - Simple Hashing with Chaining \\
		      (Deadline: 16.07.2025)
		\item Hausaufgabe 11 - Double Hashing \\
		      (Deadline: 23.07.2025)
	\end{itemize}
\end{frame}

\begin{frame}
	\textbf{Fragen?}
	\begin{itemize}
		\item Nach Übung gerne bei mir melden
		\item Tutoriumschannel oder DM an mich auf Zulip
		\item Vorlesungschannels von GAD auf Zulip (insbesondere bei Hausaufgaben)
	\end{itemize}

	\medskip
	\textbf{Feedback oder Verbesserungsvorschläge?} \\
	Gerne nach dem Tutorium mit mir quatschen oder DM auf Zulip

	\medskip
	\textbf{Bis nächste Woche!}
\end{frame}

% End Slides

\end{document}
