\documentclass{beamer}
\mode<presentation>
{
  \usetheme{ldv}
  \setbeamercovered{transparent}
}

% Uncomment this if you're giving a presentation in german...
\usepackage[ngerman]{babel}

% ...and rename this to "Folie"
\newcommand{\slidenomenclature}{Folie}


\usepackage[utf8]{inputenc}
\usepackage{amsmath,amssymb,amsfonts}
\usepackage{times}
\usepackage{graphicx}
\usepackage{fancyvrb}
\usepackage{array}
\usepackage{colortbl}
\usepackage{tabularx}

% Uncomment me when you need to insert code
\usepackage{color}
\usepackage{listings}
\usepackage{minted}
\usepackage{algpseudocode}
% End Code

\usepackage{datetime}
\usepackage{tikz}

\usetikzlibrary{calc}
\usetikzlibrary{shapes.geometric}
\usetikzlibrary{decorations.pathreplacing}
\usetikzlibrary{positioning}

% Uncomment me when you need video or sound
% \usepackage{multimedia}
% \usepackage{hyperref}
% End video

% Header
\newcommand{\zwischentitel}{Woche 8}
\newcommand{\leitthema}{Tobias Eppacher}
\newcommand{\presdatum}{\formatdate{16}{6}{2025}}
% End Header

% Titlepage
\title{Grundlagen: Algorithmen und Datenstrukturen}
\author{Tobias Eppacher}
\date{\presdatum}
\institute{School of Computation, Information and Technology}
\subtitle{Woche 8}
% End Titlepage


% Slides
\begin{document}


% 1. Slide: Titlepage
\begin{frame}
	\titlepage
\end{frame}

% 2. Slide: TOC
\begin{frame}
	\frametitle{Inhalt}
	\tableofcontents[subsectionstyle=hide]
\end{frame}

\section{Aufgaben}
\begin{frame}
	\frametitle{Aufgabe 9.1 - AVL Bebaumung}
	Gegeben sei ein AVL-Baum der nur aus einem Knoten mit Schlüssel 10 besteht. Fügen Sie
	nacheinander die Schlüssel 5, 17, 3, 1, 4 ein. Löschen Sie dann den Schlussel 4, und fügen
	Sie dann die Schlüssel 8, 2, 7, 6, 9 ein. Löschen Sie dann die Knoten mit den Schlüsseln 2,
	1, 8. Zeichnen Sie den AVL-Baum für jede Einfüge- bzw. Löschoperation und geben Sie an,
	ob Sie keine, eine Einfach- oder eine Doppelrotation durchgeführt haben.
\end{frame}

\begin{frame}[t]
	\frametitle{Aufgabe 9.1 - AVL Bebaumung}
	Insert 5:

	\bigskip
	\bigskip

	\begin{columns}
		\begin{column}{0.48\textwidth}
			\begin{center}
				\begin{tikzpicture}[nodeStyle/.style={circle, draw, minimum size=0.8cm}, scale=0.7, transform shape]
					\node[nodeStyle] (root) at (0, 0) {10};
				\end{tikzpicture}
			\end{center}
		\end{column}
		\begin{column}{0.48\textwidth}
		\end{column}
	\end{columns}
\end{frame}

\begin{frame}[t]
	\frametitle{Aufgabe 9.1 - AVL Bebaumung}
	Insert 17:

	\bigskip
	\bigskip

	\begin{columns}
		\begin{column}{0.48\textwidth}
			\begin{center}
				\begin{tikzpicture}[nodeStyle/.style={circle, draw, minimum size=0.8cm}, scale=0.7, transform shape]
					\node[nodeStyle] (root) at (0, 0) {10};
					\node[nodeStyle] (11) at (-1.5, -1.5) {5};

					\draw (root) -- (11);
				\end{tikzpicture}
			\end{center}
		\end{column}
		\begin{column}{0.48\textwidth}
		\end{column}
	\end{columns}
\end{frame}

\begin{frame}[t]
	\frametitle{Aufgabe 9.1 - AVL Bebaumung}
	Insert 3:

	\bigskip
	\bigskip

	\begin{columns}
		\begin{column}{0.48\textwidth}
			\begin{center}
				\begin{tikzpicture}[nodeStyle/.style={circle, draw, minimum size=0.8cm}, scale=0.7, transform shape]
					\node[nodeStyle](root) at (0, 0) {10};
					\node[nodeStyle](11) at (-1.5, -1.5) {5};
					\node[nodeStyle](12) at ( 1.5, -1.5) {17};

					\draw (root) -- (11);
					\draw (root) -- (12);
				\end{tikzpicture}
			\end{center}
		\end{column}
		\begin{column}{0.48\textwidth}
		\end{column}
	\end{columns}
\end{frame}

\begin{frame}[t]
	\frametitle{Aufgabe 9.1 - AVL Bebaumung}
	Insert 1:

	\bigskip
	\bigskip

	\begin{columns}
		\begin{column}{0.48\textwidth}
			\begin{center}
				\begin{tikzpicture}[nodeStyle/.style={circle, draw, minimum size=0.8cm}, scale=0.7, transform shape]
					\node[nodeStyle](root) at (0, 0) {10};
					\node[nodeStyle](11) at (-1.5, -1.5) {5};
					\node[nodeStyle](12) at ( 1.5, -1.5) {17};
					\node[nodeStyle](21) at (-2, -3) {3};

					\draw (root) -- (11);
					\draw (root) -- (12);
					\draw (11) -- (21);
				\end{tikzpicture}
			\end{center}
		\end{column}
		\begin{column}{0.48\textwidth}
		\end{column}
	\end{columns}
\end{frame}

\begin{frame}[t]
	\frametitle{Aufgabe 9.1 - AVL Bebaumung}
	Insert 4:

	\bigskip
	\bigskip

	\begin{columns}
		\begin{column}{0.48\textwidth}
			\begin{center}
				\begin{tikzpicture}[nodeStyle/.style={circle, draw, minimum size=0.8cm}, scale=0.7, transform shape]
					\node[nodeStyle](root) at (0, 0) {10};
					\node[nodeStyle](11) at (-1.5, -1.5) {5};
					\node[nodeStyle](12) at ( 1.5, -1.5) {17};
					\node[nodeStyle](21) at (-2.25, -3) {1};
					\node[nodeStyle](22) at (-0.75, -3) {3};

					\draw (root) -- (11);
					\draw (root) -- (12);
					\draw (11) -- (21);
					\draw (11) -- (22);
				\end{tikzpicture}
			\end{center}
		\end{column}
		\begin{column}{0.48\textwidth}
		\end{column}
	\end{columns}
\end{frame}

\begin{frame}[t]
	\frametitle{Aufgabe 9.1 - AVL Bebaumung}
	Remove 4:

	\bigskip
	\bigskip

	\begin{columns}
		\begin{column}{0.48\textwidth}
			\begin{center}
				\begin{tikzpicture}[nodeStyle/.style={circle, draw, minimum size=0.8cm}, scale=0.7, transform shape]
					\node[nodeStyle](root) at (0, 0) {5};
					\node[nodeStyle](11) at (-1.5, -1.5) {3};
					\node[nodeStyle](12) at ( 1.5, -1.5) {10};
					\node[nodeStyle](21) at (-2.25, -3) {1};
					\node[nodeStyle](22) at (-0.75, -3) {4};
					\node[nodeStyle](24) at ( 2.25, -3) {17};

					\draw (root) -- (11);
					\draw (root) -- (12);
					\draw (11) -- (21);
					\draw (11) -- (22);
					\draw (12) -- (24);
				\end{tikzpicture}
			\end{center}
		\end{column}
		\begin{column}{0.48\textwidth}
		\end{column}
	\end{columns}
\end{frame}

\begin{frame}[t]
	\frametitle{Aufgabe 9.1 - AVL Bebaumung}
	Insert 8:

	\bigskip
	\bigskip

	\begin{columns}
		\begin{column}{0.48\textwidth}
			\begin{center}
				\begin{tikzpicture}[nodeStyle/.style={circle, draw, minimum size=0.8cm}, scale=0.7, transform shape]
					\node[nodeStyle](root) at (0, 0) {5};
					\node[nodeStyle](11) at (-1.5, -1.5) {3};
					\node[nodeStyle](12) at ( 1.5, -1.5) {10};
					\node[nodeStyle](21) at (-2.25, -3) {1};
					\node[nodeStyle](24) at ( 2.25, -3) {17};

					\draw (root) -- (11);
					\draw (root) -- (12);
					\draw (11) -- (21);
					\draw (12) -- (24);
				\end{tikzpicture}
			\end{center}
		\end{column}
		\begin{column}{0.48\textwidth}
		\end{column}
	\end{columns}
\end{frame}

\begin{frame}[t]
	\frametitle{Aufgabe 9.1 - AVL Bebaumung}
	Insert 2:

	\bigskip
	\bigskip

	\begin{columns}
		\begin{column}{0.48\textwidth}
			\begin{center}
				\begin{tikzpicture}[nodeStyle/.style={circle, draw, minimum size=0.8cm}, scale=0.7, transform shape]
					\node[nodeStyle](root) at (0, 0) {5};
					\node[nodeStyle](11) at (-1.5, -1.5) {3};
					\node[nodeStyle](12) at ( 1.5, -1.5) {10};
					\node[nodeStyle](21) at (-2.25, -3) {1};
					\node[nodeStyle](23) at ( 0.75, -3) {8};
					\node[nodeStyle](24) at ( 2.25, -3) {17};

					\draw (root) -- (11);
					\draw (root) -- (12);
					\draw (11) -- (21);
					\draw (12) -- (23);
					\draw (12) -- (24);
				\end{tikzpicture}
			\end{center}
		\end{column}
		\begin{column}{0.48\textwidth}
		\end{column}
	\end{columns}
\end{frame}

\begin{frame}[t]
	\frametitle{Aufgabe 9.1 - AVL Bebaumung}
	Insert 7:

	\bigskip
	\bigskip

	\begin{columns}
		\begin{column}{0.48\textwidth}
			\begin{center}
				\begin{tikzpicture}[nodeStyle/.style={circle, draw, minimum size=0.8cm}, scale=0.7, transform shape]
					\node[nodeStyle](root) at (0, 0) {5};
					\node[nodeStyle](11) at (-1.5, -1.5) {2};
					\node[nodeStyle](12) at ( 1.5, -1.5) {10};
					\node[nodeStyle](21) at (-2.25, -3) {1};
					\node[nodeStyle](22) at (-0.75, -3) {3};
					\node[nodeStyle](23) at ( 0.75, -3) {8};
					\node[nodeStyle](24) at ( 2.25, -3) {17};

					\draw (root) -- (11);
					\draw (root) -- (12);
					\draw (11) -- (21);
					\draw (11) -- (22);
					\draw (12) -- (23);
					\draw (12) -- (24);
				\end{tikzpicture}
			\end{center}
		\end{column}
		\begin{column}{0.48\textwidth}
		\end{column}
	\end{columns}
\end{frame}

\begin{frame}[t]
	\frametitle{Aufgabe 9.1 - AVL Bebaumung}
	Insert 6:

	\bigskip
	\bigskip

	\begin{columns}
		\begin{column}{0.48\textwidth}
			\begin{center}
				\begin{tikzpicture}[nodeStyle/.style={circle, draw, minimum size=0.8cm}, scale=0.7, transform shape]
					\node[nodeStyle](root) at (0, 0) {5};
					\node[nodeStyle](11) at (-2.5, -1.5) {2};
					\node[nodeStyle](12) at ( 2.5, -1.5) {10};
					\node[nodeStyle](21) at (-3.75, -3) {1};
					\node[nodeStyle](22) at (-1.25, -3) {3};
					\node[nodeStyle](23) at ( 1.25, -3) {8};
					\node[nodeStyle](24) at ( 3.75, -3) {17};
					\node[nodeStyle](35) at ( 0.5, -4.5) {7};

					\draw (root) -- (11);
					\draw (root) -- (12);
					\draw (11) -- (21);
					\draw (11) -- (22);
					\draw (12) -- (23);
					\draw (12) -- (24);
					\draw (23) -- (35);
				\end{tikzpicture}
			\end{center}
		\end{column}
		\begin{column}{0.48\textwidth}
		\end{column}
	\end{columns}
\end{frame}

\begin{frame}[t]
	\frametitle{Aufgabe 9.1 - AVL Bebaumung}
	Insert 9:

	\bigskip
	\bigskip

	\begin{columns}
		\begin{column}{0.48\textwidth}
			\begin{center}
				\begin{tikzpicture}[nodeStyle/.style={circle, draw, minimum size=0.8cm}, scale=0.7, transform shape]
					\node[nodeStyle](root) at (0, 0) {5};
					\node[nodeStyle](11) at (-2.5, -1.5) {2};
					\node[nodeStyle](12) at ( 2.5, -1.5) {10};
					\node[nodeStyle](21) at (-3.75, -3) {1};
					\node[nodeStyle](22) at (-1.25, -3) {3};
					\node[nodeStyle](23) at ( 1.25, -3) {7};
					\node[nodeStyle](24) at ( 3.75, -3) {17};
					\node[nodeStyle](35) at ( 0.5, -4.5) {6};
					\node[nodeStyle](36) at ( 2, -4.5) {8};

					\draw (root) -- (11);
					\draw (root) -- (12);
					\draw (11) -- (21);
					\draw (11) -- (22);
					\draw (12) -- (23);
					\draw (12) -- (24);
					\draw (23) -- (35);
					\draw (23) -- (36);
				\end{tikzpicture}
			\end{center}
		\end{column}
		\begin{column}{0.48\textwidth}
		\end{column}
	\end{columns}
\end{frame}

\begin{frame}[t]
	\frametitle{Aufgabe 9.1 - AVL Bebaumung}
	Remove 2:

	\bigskip
	\bigskip

	\begin{columns}
		\begin{column}{0.48\textwidth}
			\begin{center}
				\begin{tikzpicture}[nodeStyle/.style={circle, draw, minimum size=0.8cm}, scale=0.7, transform shape]
					\node[nodeStyle](root) at (0, 0) {5};
					\node[nodeStyle](11) at (-2.5, -1.5) {2};
					\node[nodeStyle](12) at ( 2.5, -1.5) {8};
					\node[nodeStyle](21) at (-3.75, -3) {1};
					\node[nodeStyle](22) at (-1.25, -3) {3};
					\node[nodeStyle](23) at ( 1.25, -3) {7};
					\node[nodeStyle](24) at ( 3.75, -3) {10};
					\node[nodeStyle](35) at ( 0.5, -4.5) {6};
					\node[nodeStyle](37) at ( 3, -4.5) {9};
					\node[nodeStyle](38) at ( 4.5, -4.5) {17};

					\draw (root) -- (11);
					\draw (root) -- (12);
					\draw (11) -- (21);
					\draw (11) -- (22);
					\draw (12) -- (23);
					\draw (12) -- (24);
					\draw (23) -- (35);
					\draw (24) -- (37);
					\draw (24) -- (38);
				\end{tikzpicture}
			\end{center}
		\end{column}
		\begin{column}{0.48\textwidth}
		\end{column}
	\end{columns}
\end{frame}

\begin{frame}[t]
	\frametitle{Aufgabe 9.1 - AVL Bebaumung}
	Remove 1:

	\bigskip
	\bigskip

	\begin{columns}
		\begin{column}{0.48\textwidth}
			\begin{center}
				\begin{tikzpicture}[nodeStyle/.style={circle, draw, minimum size=0.8cm}, scale=0.7, transform shape]
					\node[nodeStyle](root) at (0, 0) {5};
					\node[nodeStyle](11) at (-2.5, -1.5) {1};
					\node[nodeStyle](12) at ( 2.5, -1.5) {8};
					\node[nodeStyle](22) at (-1.25, -3) {3};
					\node[nodeStyle](23) at ( 1.25, -3) {7};
					\node[nodeStyle](24) at ( 3.75, -3) {10};
					\node[nodeStyle](35) at ( 0.5, -4.5) {6};
					\node[nodeStyle](37) at ( 3, -4.5) {9};
					\node[nodeStyle](38) at ( 4.5, -4.5) {17};

					\draw (root) -- (11);
					\draw (root) -- (12);
					\draw (11) -- (22);
					\draw (12) -- (23);
					\draw (12) -- (24);
					\draw (23) -- (35);
					\draw (24) -- (37);
					\draw (24) -- (38);
				\end{tikzpicture}
			\end{center}
		\end{column}
		\begin{column}{0.48\textwidth}
		\end{column}
	\end{columns}
\end{frame}

\begin{frame}[t]
	\frametitle{Aufgabe 9.1 - AVL Bebaumung}
	Remove 8:

	\bigskip
	\bigskip

	\begin{columns}
		\begin{column}{0.48\textwidth}
			\begin{center}
				\begin{tikzpicture}[nodeStyle/.style={circle, draw, minimum size=0.8cm}, scale=0.7, transform shape]
					\node[nodeStyle](root) at (0, 0) {8};
					\node[nodeStyle](11) at (-2.5, -1.5) {5};
					\node[nodeStyle](12) at ( 2.5, -1.5) {10};
					\node[nodeStyle](21) at (-3.75, -3) {3};
					\node[nodeStyle](22) at (-1.25, -3) {7};
					\node[nodeStyle](23) at ( 1.25, -3) {9};
					\node[nodeStyle](24) at ( 3.75, -3) {17};
					\node[nodeStyle](33) at (-2, -4.5) {6};

					\draw (root) -- (11);
					\draw (root) -- (12);
					\draw (11) -- (21);
					\draw (11) -- (22);
					\draw (12) -- (23);
					\draw (12) -- (24);
					\draw (22) -- (33);
				\end{tikzpicture}
			\end{center}
		\end{column}
		\begin{column}{0.48\textwidth}
		\end{column}
	\end{columns}
\end{frame}

\begin{frame}
	\frametitle{Aufgabe 9.1 - AVL Bebaumung (Extra Platz)}
\end{frame}

\begin{frame}[t]
	\frametitle{Aufgabe 9.2 - ABBaumaßnahmen I}
	\scriptsize

	Führen Sie auf einem anfangs leeren (2, 3)-Baum die folgenden Operationen aus:

	$$
		\text{insert: } [19, 11, 28, 38, 37, 30, 7, 59, 41]
	$$

	gefolgt von:

	$$
		\text{remove: } [7, 11, 59, 19, 37, 41, 30, 38]
	$$

	\textbf{Hinweis:} Zeichnen Sie den Baum nach jedem Schritt. Sie dürfen in Ihrer Zeichnung auf
	Blattknoten verzichten.

	Beachten Sie außerdem das Folgende:

	\begin{itemize}
		\item Beim Aufspalten von Knoten während dem Einfügen wandert das Element am Index $\lfloor b/2 \rfloor$ nach oben.
		\item Beim Löschen von Elementen aus inneren Knoten wird üblicherweise versucht, entweder
		      den symmetrischen Vorgänger oder symmetrischen Nachfolger intelligent zu
		      wählen. Für diese Aufgabe soll darauf verzichtet werden. Stattdessen wird stets der
		      symmetrische Vorgänger verwendet. Beim Stehlen von Elementen wird zunächst der
		      linke Nachbar betrachtet. Beim Verschmelzen werden Knoten sofern möglich mit ihrem
		      linken Nachbarn, ansonsten mit dem rechten Nachbarn, vereinigt.
	\end{itemize}

\end{frame}

\begin{frame}[t]
	\frametitle{Aufgabe 9.2 - ABBaumaßnahmen I}
	\bigskip
	\bigskip

	\begin{columns}
		\begin{column}{0.48\textwidth}
			\begin{center}
				\begin{tikzpicture}[nodeStyle/.style={rectangle, draw, minimum size=0.8cm}, spaceStyle/.style={rectangle, draw, minimum width=0.2cm, minimum height=0.8cm}, scale=0.7, transform shape]
					\node[nodeStyle](root) at (0, 0) {8};
					\node[spaceStyle](root0) at (-0.5, 0) {};
					\node[spaceStyle](root1) at (0.5, 0) {};
				\end{tikzpicture}
			\end{center}
		\end{column}
		\begin{column}{0.48\textwidth}
		\end{column}
	\end{columns}
\end{frame}

\section{E-Aufgaben}
\begin{frame}
	\frametitle{E-Aufgaben}
	\begin{itemize}
		\item Aufgabe 8.5 - Rückblick: Sortierverfahren \\
		      \begin{itemize}
			      \item Laufzeitenvergleich von Sortierverfahren
		      \end{itemize}
	\end{itemize}
\end{frame}

\section{Hausaufgaben}
\begin{frame}
	\frametitle{Hausaufgaben}
	\begin{itemize}
		\item Hausaufgabe 6 - Sortierende Heaps \\
		      (Deadline: 18.06.2025)
		\item Hausaufgabe 7 - Binomial Heap \\
		      (Deadline: 25.07.2025)
		\item Hausaufgabe 8 - AVL Baum \\
		      (Deadline: 02.07.2025)
	\end{itemize}
\end{frame}

\begin{frame}
	\textbf{Fragen?}
	\begin{itemize}
		\item Nach Übung gerne bei mir melden
		\item Tutoriumschannel oder DM an mich auf Zulip
		\item Vorlesungschannels von GAD auf Zulip (insbesondere bei Hausaufgaben)
	\end{itemize}

	\medskip
	\textbf{Feedback oder Verbesserungsvorschläge?} \\
	Gerne nach dem Tutorium mit mir quatschen oder DM auf Zulip

	\medskip
	\textbf{Bis nächste Woche!}
\end{frame}

% End Slides

\end{document}
